\section{Threats to Validity}
\label{THREATS TO VALIDITY}

This section discusses the study's threats to validity following the classification proposed by Wohlin et al. ~\cite{wohlin2012experimentation} and the strategies applied to mitigate them.

\textit{Internal validity:} We applied the questionnaire during the period that the company made available for the research. This session lasted around 50 minutes, which may have influenced the results due to fatigue. Another threat is related to understanding each of the psychometric instruments used in the study. The first and second authors ran a training session with the study's subjects to mitigate this threat. \textit{Conclusion validity:} We obtained the results from the data using the Kendall correlation coefficient. We also adopted a free software for statistical computing and interpreted the correlation coefficients using psychology guidelines proposed in \cite{dancey2007statistics}. 

\textit{Construct validity:} We used psychometric instruments previously validated by other studies and mapped 16PF aspects, BFI types, and personality factors CC based on the literature. However, the translation into Portuguese has removed essential aspects of measurement for the CC. The SE scenarios can be regionalized, not matching the scenarios of the participants in this study. \textit{External validity:} the relatively small sample size could limit external validity. Therefore, this study should be replicated with a larger sample size to confirm the initial results and address external validity issues. Lastly, the generalizability of these results is subject to certain limitations.

