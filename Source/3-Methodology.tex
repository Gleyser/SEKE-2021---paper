\section{Research Methodology}
\label{Empirical Study}

This section presents the research methodology for the empirical study, including research questions, subjects' profiles, experimental materials, and data analysis procedures.

\textbf{Objective and Research Questions}. Our study aimed to analyze and evaluate the correlation between three state-of-art psychometric instruments (i.e., BFI, 16PF, and CC) in the context of software development activities. Given this, we formulated some Research Questions (RQs):

\begin{itemize}
    \item {\bfseries RQ1:} To what extent do BFI dimensions correlate with 16PF  considering software developers' personality?
    \item {\bfseries RQ2:} To what extent do 16PF aspects correlate with CC factors considering software developers' personality?
    \item {\bfseries RQ3:} To what extent do BFI dimensions correlate with CC factors considering software developers' personality?
 \end{itemize}
 
\textbf{Subjects}. We applied the psychometric instruments with 29 software developers (24 men and five women) from one Brazilian software organization. This organization had approximately 250 employees and produced more than 40 projects in collaboration with multinational partners. The employees were organized into small agile teams (around five to ten members).

The participant's ages ranged from 21 to 29 years, with a mean of 24 years. They had an average of three years of experience with software development. They developed Web and mobile applications using different technologies (e.g., Javascript, Java, HTML). Overall, the subjects' profile meets our study assumptions since all of them work in software development.

\textbf{Experimental Materials}. We created a questionnaire \footnote{Available in https://bit.ly/38E6yhP} with five sections to apply the psychometric instruments and submitted it to the ethics committee from the Federal University of Campina Grande (UFCG) for analysis and approval before conducting the study. The ethics committee gave us the certification (02505718.0.0000.5182), meaning we could collect data using the questionnaire. In the following, we present information about the questionnaire.

The \textit{first section} of the questionnaire presented the study objectives and the consent form, as approved by the ethics committee. This section contained information to motivate the participants to participate in the study. The \textit{second section} contained questions to collect demographic data, including name, gender, experiences, and age. The \textit{third section} contained BFI-44 questions translated and adapted to Portuguese by Andrade~\cite{andrade2008evidencias}. We also consulted the BFI-44 in Portuguese available on the Berkeley Personality Lab site\footnote[2]{https://www.ocf.berkeley.edu/$\sim$johnlab/bfi.htm}. All questions were answered through a five-point Likert-type scale: 1 (Strongly disagree) - 5 (Strongly agree), expressing their agreement degree regarding the question's descriptions. The \textit{fourth section} contained 16PF questions. The 16PF has about 60 questions (statements), each of them to be answered through a seven-point Likert-type scale (from ``agree'' to ``disagree''). 

Finally, the \textit{fifth section} included the CC questions. We translated and adapted the Context Cards to Portuguese. This psychometric instrument includes 60 cards (e.g., situations, questions), in which each card presents a situation and two optional answers described by A and B. The optional answers describe SE situations. One example of the original text on the one card is presented next. Situation (``During a team base discussion...''), Option A (``Evidence suggests that what we learn is mostly from our conflicts.''), and Option B (``I believe compromise between people for a common ground more successful.''). Each person spends about 40 minutes - on average - to complete the questionnaire.

\textbf{Procedure}. We applied the questionnaire during the period that the software organization made available for the research. Before the questionnaire application, the study's authors ran a training session with the participants. The training session lasted 30 minutes, and the participants spent between 30 to 40 minutes answering the entire questionnaire. The training session aimed to present the questionnaires' concepts and level the participants' understanding regarding the psychometric instruments and study's goals. In other words, we motivated the participants to answer based on reality (not intentions) and understood the questions. Further, we emphasized that there were no ``correct'' answers and that they would not be identified. Such support minimized internal validity threats. 

\textbf{Analysis Procedure}. We defined the following criteria for comparing the psychometric instruments: the number of questions, option type, response time, and the correlation between the results (i.e., the correlation between the facets, dimensions, and aspects of psychometric instruments). The correlation analysis sought to verify the correlation between the personality traits presented by instruments for each participant. The BFI-44, 16PF, and Context Cards instruments contain different sets of Likert-type items combined into single composite scores (as explained in Section~\ref{BACKGROUND}). Thus, they calculate a score for each facet or dimension of the psychometric instrument. Each of these composite scores provides a quantitative measure of a personality trait. 

Considering that the variables measured in this study are ordinal, we used the Kendall correlation coefficient. Kendall correlation coefficient is a nonparametric measure of the strength and direction of the association between two variables measured on at least an ordinal scale~\cite{puth2015effective}. We also used the correlation coefficients interpretation for psychology described by Dancey and Reidy \cite{dancey2007statistics}.













