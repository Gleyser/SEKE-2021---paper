\section{Background and Related Work}
\label{BACKGROUND}

This section presents an overview of psychometric instruments and describes the studies that compared psychometric instruments in the context of Software Engineering.

\textbf{Psychometric Instruments Overview}. Personality involves different theoretical perspectives, definitions, and levels of abstraction. We used the personality definition established by Ryckman~\cite{ryckman2012theories}, in which it is defined as ``a dynamic and organized set of characteristics possessed by a person that uniquely influences his or her cognitions, motivations, and behaviors in various situations''. We used this definition due to its popularity in Software Engineering research. 

Psychometric instruments have been used to evaluate personality \textit{traits} and \textit{types} of individuals, usually by using questionnaires. Personality traits are stable characteristics of individuals such as being optimistic, sociable, and imaginative, whereas Personality types are constructs that indicate independent groups such as mediator, entrepreneur, and adventurer. For instance, using the 16PF psychometric instrument, a person that scores high on the traits ``introversion'', ``sensing'', ``thinking'', and ``judging'' would be classified as being of the personality type ``logistician''.

\textbf{Psychometric Instruments in SE}. Software development organizations use psychometric instruments to measure their members' personality~\cite{lounsbury2007investigation, wyrich2019theory}. The most used psychometric instruments in SE are based on the Big Five (BF) theory - e.g., Big Five Inventory (BFI) \cite{cruz2015forty}. Other psychometric instruments have been widely used in SE research, such as the 16 Personality Factor Questionnaire (16PF) and Context-specific survey instrument (CC). Next, we describe each of the aforementioned psychometric instruments.

\textit{Big Five Inventory (BFI)} describes the personality by employing broad factors (dimensions) of personality traits~\cite{costa2010neo}. Its five dimensions are Extraversion, Agreeableness, Conscientiousness, Neuroticism (a.k.a. Emotional Stability), and Openness to Experience (sometimes called Intellect or Imagination). The BFI-44 is a self-report inventory created to measure the Big Five dimensions. The psychometric instrument contains 44 items and consists of short and descriptive phrases that respondents rated on a 5-point scale ranging from strong disagreement to strong agreement. This method is not in the public domain. However, it is readily available for researchers to use for non-commercial research purposes. 

\textit{The 16 Personality Factor Questionnaire (16PF)} is a psychometric instrument to identify characteristics, personality traits, and behavior. 16PF was published in 1949 and has been used to evaluate personality in many contexts, including career assessment and SE \cite{gomesevaluating}. The 16PF online test comprises five personality dimensions, making up 16 personality types. 16PF has five aspects: Mind, Energy, Nature, Tactic, and Identity. 16PF generates 16 types of personalities by acronyms generated from the dichotomies emitted by the psychometric instrument's aspects. The combination of four personality aspects results in a personality type. For instance, the combination of Extroversion (E), Observant (S), Thinking (T), Judging (J), and Assertive (-A) result in the personality ESTJ-A.  The Identity scale (i.e., assertive or turbulent) is in all personality types because it affects other scales. As a result, when we considered this scale, the method describes 32 different personality types. 

\textit{Context-specific survey instrument (CC)} was proposed by Yilmaz et al.~\cite{yilmaz2017examination} aims to reveal and illustrate the personality characteristics of the individuals in software development teams. The instrument combines situations from companies with basic patterns (items) of the Big Five Inventories (BFI-44) questionnaire to create a card game-based personality identification method~\cite{yilmaz2017examination}. The context cards describe the human personality traits in terms of five fundamental factors: Extraversion, Openness, Agreeableness, Neuroticism, and Conscientiousness.  This model identified six themes (traits) for each factor, totaling 30 themes. For example, the factor Extroversion has the traits talkative, assertive, energetic, active, approachable, and outgoing.
 
 \textbf{Psychometric Instruments Evaluation in SE}. Next, we discuss studies that assessed psychometric instruments in SE. Jia et al.~\cite{jia2015comparative} reviewed and compared three psychometric instruments (i.e., BFI, MBTI, and KTS). They observed the number of questions, option type, and time spent answering the test. The researchers collected empirical evidence for comparison from articles published between 2010 and 2014. They concluded that BFI is the more suitable alternative to evaluate soft-skills in software development activities. 

Another study by Gulati et al.~\cite{gulati2015comparative} compared studies published between 2003 and 2014 that analyzed human factors in software engineering. They concluded that the most popular psychometric instruments in SE are MBTI and BFI. Finally, Balijepally et al.~\cite{balijepally2006assessing} dedicated their research mainly to compare BFI and MBTI. They suggested that BFI is more valuable than MBTI because BFI provides better measures for all MBTI factors, and it also evaluates Neuroticism, an important personality trait. 

Even though these studies promote a great discussion relating to software development activities (i.e., SE area) and soft-skills (i.e., Psychology area), they perform the models' comparison based solely on data available in the literature. Consequently, we conclude that these studies highlight the importance of human psychology in SE and realized conceptual comparisons but lack evidence about how the psychometric instruments compare in the context of SE. To address this gap, we compared personality instruments by collecting data from software developers and analyzing the correlation between the instruments' answers.