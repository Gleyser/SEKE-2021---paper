\section{Introduction}

The success of software development projects is directly related to the team members’ technical (a.k.a. Hard skills) and non-technical skills (a.k.a Soft Skills) \cite{canedo2019factors}. Soft skills are becoming more important in the industrial environment because they affect team cohesion and team climate~\cite{gilal2017effective, ruiz2019understanding}, impacting its productivity and outcomes' quality~\cite{gomesevaluating, Iqbal2019BigfivePT}. 

A key aspect of studying soft skills is personality. Many studies investigated the effects of personality on teamwork performance in the last forty years~\cite{cruz2015forty, soomro2016effect}. These studies used several psychometric instruments to evaluate personality types and personality traits of software engineers~\cite{yilmaz2017examination, cruz2015forty, soomro2016effect}. The studies mostly used the Myers-Briggs Type Indicator (MBTI) or tests based on the Big Five (BF), such as Big Five Inventory (BFI) and Revised NEO Personality Inventory (NEO-PI-R). However, other psychometric instruments were also used, such as International Personality Item Pool (IPIP), the Sixteen Personality Factor Questionnaire (16PF), and Context-Specific Survey Instrument (Context Cards) \cite{cruz2015forty}. %\footnote[]{Corresponding author: gleyser.guimaraes@virtus.ufcg.edu.br}

Choosing a psychometric instrument is not straightforward because some require training and a license to be used. Moreover, McDonald and Edwards reported misuse of personality tests in software engineering (SE)~\cite{mcdonald2007should}. They argued that the inappropriate use of psychological tests and fundamental misunderstandings of personality theory caused a lack of progress in this field. Additionally, Graziotin et al. \cite{graziotin2020behavioral} demonstrated a deeper confusion on assessing related constructs using personality tests. For example, Capretz and Ahmed~\cite{capretz2010making} considered that the introvert trait is suitable for the programmer role, whereas Gorla and Lam~\cite{gorla2004should} concluded that it is the Extrovert trait. 

Having reliable data is the most critical factor for any SE measurement approach. Such observation is also valid for psychometric instruments. The inadequate application of psychometric instruments and their interpretation might lead to invalid results, economic loss, and harm to individuals. If the psychometric instruments or their usage are not valid, all the resulting conclusions are also invalid. Psychometric instruments measure latent variables (i.e., unobservable constructs) such as intelligence, personality, and happiness. Therefore, evaluating the psychometric instrument used is crucial to ensure that the variables are measured correctly.

Cruz et al.~\cite{cruz2015forty} discusses the existence of many disagreements in the SE research community regarding (i) the application of psychometric instruments and (ii) the interpretation of their results, comparing psychometric instruments' constructs to understand how to measure software engineers' personalities and their impact on productivity and quality. However, no other works are performing a similar comparison analysis in the past five years. 

Moreover, Gulati et al.~\cite{gulati2015comparative} examined studies based on human factors in software engineering. They compared studies relating to different personality instruments (i.e., MBTI, KTS, and BFI). Balijepally et al.~\cite{balijepally2006assessing} focused their research on comparing two emerging models, BFI and MBTI, for assessing personality traits in SE. To the best of our knowledge, these studies promote a discussion relating to software engineering and psychology but do not explore the correlation analysis among the personality instruments.

To address this gap, we investigated the similarity between three psychometric instruments: Big Five Inventory (BFI), 16 Personality Factors (16PF), and Context Cards (CC). We used these instruments because (i) studies in SE recurrently use them, (ii) they are of the public domain or readily available for researchers, (iii) they are clear to use in SE, and their data analysis have a vocabulary that is easy to understand, (iv) they do not have a large number of items, thus easing its execution and interpretation. We collected data from 29 subjects with about three to five years of experience in the area, and most of them were developers of web or mobile projects. We compared the instruments in terms of the number of questions, option type, response time, and the Kendall correlation.

This paper details the applied study and summarizes the results regarding the similarity of the evaluated psychometric instruments. The remainder of this paper is organized as follows.  Section~\ref{BACKGROUND} provides a background with an overview of psychometric instruments for SE. Section~\ref{Empirical Study} describes the study design. Section~\ref{Results and Discussion} presents the results and discusses the answers to the research questions.  Section \ref{THREATS TO VALIDITY} analyzes the study's threats to validity. Finally, Section~\ref{CONCLUSIONS AND FUTURE WORKS} presents our conclusions and directions to future work.











