\documentclass[conference]{IEEEtran}
%\documentclass[times,10pt,twocolumn]{article}
  %\pdfpagewidth=8.5truein
  %\pdfpageheight=11truein
%\renewcommand{\baselinestretch}{0.96}

\usepackage{booktabs} % For formal tables
\usepackage{latex8}
\usepackage[table,xcdraw]{xcolor}
% Copyright
%\setcopyright{none}
%\setcopyright{acmcopyright}
%\setcopyright{acmlicensed}
%\setcopyright{rightsretained}
%\setcopyright{usgov}
%\setcopyright{usgovmixed}
%\setcopyright{cagov}
%\setcopyright{cagovmixed}

%\usepackage[portuges,english,brazil]{babel}
\usepackage[T1]{fontenc}
\usepackage[utf8]{inputenc}
\usepackage{enumerate}
\usepackage{enumitem}
\usepackage{multirow}
\usepackage[pdftex]{graphicx}
\usepackage{comment}
\usepackage{enumitem}
\usepackage{graphicx}
\usepackage{times}
\usepackage{url}
\usepackage{authblk}
\usepackage{tikz}
\usepackage{hyperref}
\hypersetup{colorlinks=true,allcolors=black}
\usepackage{hypcap}
\usepackage{amsmath}
\usepackage{graphicx}

% DOI
%\acmDOI{10.475/123_4}

% ISBN
%\acmISBN{123-4567-24-567/08/06}

%Conference
%\acmConference[SBES'31]{31st Brazilian Symposium on Software Engineering}{September 18 – 22, 2017}{\\Fortaleza, Ceará, Brazil} 
%\acmYear{2017}
%\copyrightyear{2017}

%\acmPrice{15.00}
\hyphenation{op-tical net-works semi-conduc-tor}

\newcommand\copyrighttext{%
  \footnotesize Reserved for possible DOI number \hfill }
\newcommand\copyrightnotice{%
\begin{tikzpicture}[remember picture,overlay]
\node[anchor=south,yshift=33pt] at (current page.south)
{\parbox{\dimexpr\textwidth\relax}{\copyrighttext}};
\end{tikzpicture}%
}

\renewcommand{\baselinestretch}{0.95}

\begin{document}
\title{A Comparative Analysis of Alalalalalalalalagile Teamwork Quality Measurement Instruments}

\author{M. Silva, A. Silva, M. Perkusich, D. Albuquerque, E. Guimarães, H. Almeida, A. Perkusich and K. Costa}
\affiliation{Intelligent Software Engineering Group (ISE/VIRTUS), Federal University of Campina Grande, Campina Grande, Paraiba, Brazil \\
\{manuel.silva, arthur.silva, mirko, danyllo.albuquerque, hyggo, perkusic, angelo, kyller\}@virtus.ufcg.edu.br \\
\The Pennsylvania State University, Malvern, Pennsylvania, USA. \\
\ ezt157@psu.edu
}
\maketitle
\copyrightnotice

\begin{abstract}


\end{abstract}

%
% The code below should be generated by the tool at
% http://dl.acm.org/ccs.cfm
% Please copy and paste the code instead of the example below. 
%
%\begin{CCSXML}
%<ccs2012>
%<concept>
%<concept_id>10002944.10011123.10011124</concept_id>
%<concept_desc>General and reference~Metrics</concept_desc>
%<concept_significance>300</concept_significance>
%</concept>
%</ccs2012>
%\end{CCSXML}

%\ccsdesc[300]{General and reference~Software Metrics}

% We no longer use \terms command
%\terms{Theory}

\begin{IEEEkeywords} Teamwork; Agile Software Development; Agile; Bayesian Networks; Structural Equation Modeling; Comparison; Measurement; Mean Relative Error \end{IEEEkeywords}

\section{Introduction}
\label{sec:intro}

\section{Background}
\label{sec:background}

This section describes information related to TWQ-SEM (Section~\ref{subsec:sem}) and TWQ-BN (Section~\ref{subsec:bayesian_network}), and the mapping carried out between the instruments' variables (Section~\ref{subsec:study_variables}).

\subsection{Structural Equation Modeling Based Instrument (TWQ-SEM)}
\label{subsec:sem}

Hoegl and Gemuenden~\cite{hoegl2001teamwork} conducted an empirical study to assess how TWQ influences the project's and team's success. For that, they proposed a TWQ evaluation model composed of a high-order factor (dependent variable) and six TWQ facets (independent variables). The evaluation of the influence of the TWQ on the success of the project and the team were analyzed using the Structural Equation Modeling (SEM). 

As a data source to feed their model, they used a questionnaire with 61 questions. Each variable was associated with between three and ten questions, which were answered on a 5-point Likert scale. The results of the study showed that: (i) TWQ is significantly associated with team performance; (ii) TWQ has a strong correlation with the personal success of team members; and (iii) TWQ provides a comprehensive measure of the collaborative team task process, focusing on quality interactions.

Lindsjørn \textit{et al}.~\cite{lindsjorn2016teamwork} replicated Hoegl and Gemuenden's ~\cite{hoegl2001teamwork} study in the context of Agile Software Development (ASD). Their goal was to assess whether the TWQ-SEM (and its associated questionnaire) is feasible for ASD. Thus, they verified empirically whether the TWQ-SEM produced a covariance matrix consistent with the sample's covariance matrix.

Figure~\ref{} presents Lindsjørn \textit{et al}.'s~\cite{lindsjorn2016teamwork} results. It only considers the variables directly related to the theoretical construct of TWQ, ignoring the ones related to the project's and team's success. The arrows represent the standardized factorial loads for each construct. It shows the variation explained by the variable in the TWQ. In the SEM approach, as a general rule, factorial load 0.7 or higher represents that the factor extracts sufficient variation from this variable.

In addition, figure~\ref{} shows that some factors (i.e., cohesion, communication, and mutual support) have a strong influence on TWQ. The data source for TWQ-SEM is a questionnaire, where the latent variables (LV) are considered metrics, and each one is linked to the questions in the questionnaire. However, the relation between questions to a given LV in TWQ-SEM is many-to-one. Given this, the authors defined that the value for a given LV is the arithmetic mean of the responses for the set of questions related to it.

\subsection{Bayesian Networks Based Instrument (TWQ-BN)}
\label{subsec:bayesian_network}

This instrument was proposed by Freire et al.~\cite{freire2018bayesian}, being based on a Bayesian network to assess and improve the Teamwork Quality in ASD. They built the instrument following two stages. First, they identified the variables that make up the TWQ construct based on literature analysis. Then, they mapped the relationships between the variables, defining the Bayesian network's structure, and quantified the relationships, defining the Bayesian network's probability functions. To this end, they relied on the knowledge of experts. Figure~\ref{} presents the resulting Bayesian network.

Freire et al.~\cite{freire2018bayesian} assessed the proposed model and procedure for applying it to the team's routines through a case study with one company. The study's results indicated that: (i) the model helped to identify opportunities for team improvement and to assess the Teamwork quality; (ii) the cost-benefit of using the process application model was positive; and (iii) the process was easy to be learned and implemented in the team's routine.

In practical terms, a stakeholder (e.g., project manager or technical leader) can use the proposed Bayesian network for diagnosis or prognosis. The \textit{diagnosis} is enabled by Bayesian networks' capabilities of back-propagation given the d-separation rules and treating all variables as observable~\cite{anderson2004causal} (that is, users can enter data in any of the model's variables). Thus, stakeholders can diagnose Teamwork or any of its facets by inserting evidence into the node~\textit{Teamwork} or any other non-input node. Consequently, the remaining nodes' values are updated through back-propagation, which the users can use to execute the "what-if'' analysis and support management decisions on improving TWQ.

As mentioned above, the instrument can also be for prognosis (or prediction). To perform \textit{prognosis}, the stakeholder enters data into input nodes (that is, evidence) and, through forward propagation following the d-separation rules, the remaining nodes' values are updated. Prognosis is the most usual use case. In this case, the input nodes can be considered measures and linked to data sources (e.g., a questionnaire or Computer-Aided Software Engineering tools). As TWQ-SEM, the TWQ-BN is associated with a questionnaire. In practice, for prognosis purposes, the data is entered into TWQ-BN by answering its associated questionnaire. Section~\ref{subsec:analysis} presents how we operationalized feeding TWQ-BN with the data collected in our research; especially, discussing the case of having multiple questionnaire responses per team.

\subsection{Equivalence between the variables of the TWQ instruments}
\label{subsec:study_variables}

To perform the comparative analysis between the models, we mapped the variables between them, taking into account the similarity of the authors' definitions. Note that TWQ-BN has more variables than TWQ-SEM. While TWQ-BN contains 17 variables, TWQ-SEM contains only seven. Since TWQ-SEM is the one with the least variables, we try to map its variables with TWQ-BN. 

In some cases where the variables have the same name and definition, we map directly; This is the case with \textit{Communication}, \textit{Coordination}, \textit{Cohesion}, and \textit{Teamwork}. In other cases, the variables have different names but similar definitions. We mapped the variable \textit{Balance of member contribution}, from TWQ-SEM, to  \textit{Team orientation}, from TWQ-BN. We did this because, by looking at the TWQ-BN variables individually, we realized that it was not possible to \textit{Balance of member contribution} to a single TWQ-BN variable. However, we believe that this association could be made with the union of the variables \textit{Personal attributes}, \textit{Expertise}, and \textit{Team Orientation} of the TWQ-BN Team. Given that, \textit{Team Orientation} is the child of \textit{Personal Attributes} and \textit{Expertise}.

The variables \textit{Effort}, from TWQ-SEM and \textit{Collaboration}, from TWQ-BN, were mapped as both reflect team members' willingness to achieve the team's goals. \textit{Effort} is related to sharing the workload and prioritizing the teams' tasks, while \textit{Collaboration} is about the commitment that the team members maintain among themselves to achieve common goals. Finally, the variables \textit{Mutual Support}, from TWQ-SEM to \textit{Self-Organazing}, from TWQ-BN, were mapped since both describe team members' ability to organize themselves to achieve common goals.

After mapping all the variables from both instruments, we noticed that some variables from TWQ-BN had no similarity with other variables in TWQ-SEM. \textit{Team Autonomy} is identified by Eloranta et al.~\cite{Eloranta2016194} as a critical factor to keep the team motivated but is not addressed in TWQ-SEM. \textit{Daily meetings}, which is a prevalent work synchronization practice adopted by the majority of agile teams~\cite{stray2016daily}, is also not addressed in TWQ-SEM. At this point, we were not concerned with the variables without mapping because this was expected since the models have a different number of variables.

\section{Research Methodology}
\label{sec:methodology}

This section presents the research methodology applied for the investigation and comparison analysis of two agile TWQ models: TWQ-BN and TWQ-SEM. Next, we discuss the research questions and the study's hypotheses (Section~\ref{subsec:research_questions}, subjects profile (Section~\ref{subsec:subjects}, instrumentation (Section~\ref{subsec:instrumentation}, and data collection and analysis procedures (Sections~\ref{subsec:data_collection} and~\ref{subsec:analysis}, respectively).

\subsection{Research Questions}
\label{subsec:research_questions}

This study aimed to compare TWQ-BN and TWQ-SEM results in terms of how each instrument measure their variables. Given this, to guide our empirical investigation, we defined the following research question:

\textbf{RQ} -  To what extent are the results of TWQ-BN and TWQ-SEM equivalent? \newline \newline For the research question, we derived the null and alternative hypotheses. For simplicity, Table~\ref{tab:hypotheses} only includes the null hypotheses. 

To address the research question, we investigated different variables related to communication, coordination, cohesion, effort, mutual support, and balance of member contribution. Therefore, a null hypothesis was created for each variable, which corresponds to hypotheses \textbf{H1} to \textbf{H1.f}. For the hypotheses assessment, we applied statistical analysis to the data obtained in each of the variables mapped between the models using the Pearson's correlation coefficient \cite{}. To answer \textbf{RQ}, we conducted an empirical study with members from two software development companies.

\begin{table}
\centering
\caption{ Study Hypotheses }
\label{tab:hypotheses}
\begin{tabular}{cc} 
\hline
\rowcolor{black} \textbf{\textcolor{white}{H}}  & \textbf{\textcolor{white}{Description}}                                                                                                                                                       \\ 
\hline
H\_1.0                                          & \begin{tabular}[c]{@{}c@{}}The measurement models from both instrument yield \\equivalent results\end{tabular}                                                                                \\
\rowcolor[rgb]{0.753,0.753,0.753} H\_1a.0       & \begin{tabular}[c]{@{}>{\cellcolor[rgb]{0.753,0.753,0.753}}c@{}}The~measurement model from both instrument yield \\equivalent results for the variable \textit{Communication} \end{tabular}  \\
H\_1b.0                                         & \begin{tabular}[c]{@{}c@{}}The~measurement model from both instruments yield \\equivalent results for the variable \textit{Coordination} \end{tabular}                                       \\
\rowcolor[rgb]{0.753,0.753,0.753} H\_1c.0       & \begin{tabular}[c]{@{}>{\cellcolor[rgb]{0.753,0.753,0.753}}c@{}}The~measurement model from both instruments yield \\equivalent results for the variable \textit{Cohesion} \end{tabular}      \\
H\_1d.0                                         & \begin{tabular}[c]{@{}c@{}}The~measurement model from both instruments yield \\equivalent results for the variable \\\textit{Balance of Member Contribution} \end{tabular}                   \\
\rowcolor[rgb]{0.753,0.753,0.753} H\_1e.0       & \begin{tabular}[c]{@{}>{\cellcolor[rgb]{0.753,0.753,0.753}}c@{}}The~measurement model from both instruments yield \\equivalent results for the variable \textit{Effort} \end{tabular}        \\
H\_1f.0                                         & \begin{tabular}[c]{@{}c@{}}The~measurement model from both instruments yield \\equivalent results for the variable \textit{Mutual Support} \end{tabular}                                    
\end{tabular}
\end{table}

\subsection{Subjects}
\label{subsec:subjects}

We collected data from the subjects worked at two software development companies. From here on, we refer to them as Organization A and Organization B. We chose these companies given academic-industrial relations. These companies carries out research, development, and technological innovation projects with industry partners. We collected data from $25$ software development teams, $24$ from Organization A, and one from Organization B. All the teams managed their projects using Scrum, and each project applied software development practices given their domain, which included embedded, mobile, Web, and data-driven apps. They held all Scrum ceremonies, including the Daily Scrum, Sprint Planning, Sprint Review, and Sprint Retrospective. For the length of the Sprints, they lasted between two to three weeks. 

Moreover, we interviewed $162$ members of these teams in different roles (i.e., developers, quality analysts (QA), technical leaders (TL), Scrum Masters (SM), and managers (M)). Managers take the lead in all phases and activities, including project planning, management, monitoring, and closing. They were responsible for communication between the customer and the Scrum Team. SMs acted as facilitators and coaches for teams and were responsible for causing the removal of impediments. The TLs were closer to the managers and were responsible for ensuring that the products were delivered on time and within the specified quality. Both SMs and TLs performed the work to deliver the product, together with the developers and QAs. Figure~\ref{fig:subjects_information} show more information about the subjects and the composition of their respective teams. See Supplementary Material for more details. 

\subsection{Instrumentation}
\label{subsec:instrumentation}

Aims to enable the data collection, we prepared an online questionnaire with three sections.  The \textbf{first} section contained questions to collect demographic data, The \textbf{second} section contained questions related to the TWQ-BN; we used the same questionnaire proposed by Freire et al.~\cite{freire2018bayesian}. Altogether, the questionnaire had nine questions, one for each TWQ-BN's input nodes. Finally, the \textbf{third} section contained questions related to TWQ-SEM. For this purpose, we adapted the questionnaire presented by Lindsjørn et al.~\cite{lindsjorn2016teamwork}, including only questions related to the TWQ construct, resulting in $38$ questions. 

To calculate the inferences for TWQ-BN, we used AgenaRisk\footnote{\url{https://www.agenarisk.com/}} because it was the same tool used by Freire et al.~\cite{freire2018bayesian}. For TWQ-SEM, we used RStudio\footnote{\url{https://rstudio.com/}}. We used RStudio due to the ease of use and available functions, which reduced the operational costs. To perform correlation between instruments' variables, we used Colab\footnote{\url{https://colab.research.google.com/}}. Moreover, we used online spreadsheets to manage the questionnaires' answers and store the data regarding the instruments' execution. The data collected and spreadsheets are online available in the Supplementary Material.

\subsection{Data Collection Procedure}
\label{subsec:data_collection}

To collect the data, we conducted interview sessions with the subjects who volunteered and had the companies' consent to participate. The sessions were held in the rooms where the participants performed their activities and took place right after their Sprint retrospective meetings. We selected this moment because the respondents had just reflected on the outcomes of the previous Sprint. 

Before applying the questionnaire, the first author of this article held a training session with the subjects. The training session lasted ten minutes and aimed to (i) motivate participants to answer the questions based on reality (not intentions),  (ii) leveling the participants' knowledge about the TWQ instruments and (iii) clarify the understanding of the questions that make up the questionnaire. Soon after, subjects were asked to answer the questionnaire that was made available online. A researcher was present to avoid the interviewees exchanging information between them, and interrupt if they noticed that the respondents were tired. Twenty-five data collection sessions were carried out (One for each team), with an average duration of $40$ minutes.

\subsection{Data Analysis Procedure}
\label{analysis}

\section{Results}
\label{sec:results_discussion}

% Jogar a imagem já com os valores. 2x3
% Varificar como os resultados sao apresentados para correlação.

\section{Discussion}
\label{sec:discussion}

\section{Threats to validity}
\label{sec:threats}

This section presents the study's threats and the strategies applied to minimize them. For doing so, we followed the classification schema proposed by Wohlin et al.~\cite{wohlin2012experimentation}.

\textbf{Internal validity} - In each interview session, respondents were asked to answer the questionnaire with questions regarding the two instruments. Each session lasted an average of $40$ minutes, which may have influenced the results due to fatigue. To minimize this, all respondents responded simultaneously, while the first author followed the process. To ensure an understanding of research objectives and both instruments' questions, the interviewees received a training session. Also, perhaps the questions were not clear enough, to mitigate this, we used the questionnaires proposed by Freire et al.~\cite{freire2018bayesian} and Lindsjørn et al.~\cite{lindsjorn2016teamwork} that the authors previously validated.

\textbf{External validity} - All teams that participated in the research used Scrum as a management methodology. Although this can limit the generalization of scrum teams' results, Scrum is present among the five most-used agile practices in the industry. According to the latest report by State of Agile~\footnote{\url{https://bit.ly/2FMfvdp}}, $72\%$ of projects use Scrum. 

\textbf{Conclusion validity:} analyzing the equivalence between the measurements of the models using only the Pearson's correlation coefficient can generate false positives. To minimize this, we also analyzed the results using the 95\% confidence interval.

\textbf{Construct validity:} we combined the perceptions of individuals across all teams using the arithmetic mean. Despite knowing that this is a controversial topic, we chose to follow the procedure used by Lindsjørn et al.~\cite{lindsjorn2016teamwork}. Also, we perform the mapping of variables between models. Although TWQ-BN has more variables than TWQ-SEM, the variables were mapped by similarity in their definitions.

\section{Conclusions and Future Works}
\label{sec:conclusion}

\section{Acknowledgment}
\label{sec:acknowledgment}

This study was financed in part by the Coordenação de Aperfeiçoamento de Pessoal de Nível Superior – Brasil (CAPES) – Finance Code 001.

\bibliographystyle{IEEEtran}
\bibliography{sigproc} 

\end{document}
